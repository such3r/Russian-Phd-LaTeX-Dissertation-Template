
{\actuality} Обзор, введение в тему, обозначение места данной работы в
мировых исследованиях и~т.\:п., можно использовать ссылки на другие
работы~\cite{Gosele1999161} (если их нет, то в автореферате
автоматически пропадёт раздел <<Список литературы>>). Внимание! Ссылки
на другие работы в разделе общей характеристики работы можно
использовать только при использовании \verb!biblatex! (из-за технических
ограничений \verb!bibtex8!. Это связано с тем, что одна и та же
характеристика используются и в тексте диссертации, и в
автореферате. В последнем, согласно ГОСТ, должен присутствовать список
работ автора по теме диссертации, а \verb!bibtex8! не умеет выводить в одном
файле два списка литературы).

Для генерации содержимого титульного листа автореферата, диссертации и
презентации используются данные из файла \verb!common/data.tex!. Если,
например, вы меняете название диссертации, то оно автоматически
появится в итоговых файлах после очередного запуска \LaTeX. Согласно
ГОСТ 7.0.11-2011 <<5.1.1 Титульный лист является первой страницей
диссертации, служит источником информации, необходимой для обработки и
поиска документа.>> Наличие логотипа организации на титульном листе
упрощает обработку и поиск, для этого разметите логотип вашей
организации в папке images в формате PDF (лучше найти его в векторном
варианте, чтобы он хорошо смотрелся при печати) под именем
\verb!logo.pdf!. Настроить размер изображения с логотипом можно в
соответствующих местах файлов \verb!title.tex!  отдельно для
диссертации и автореферата. Если вам логотип не нужен, то просто
удалите файл с логотипом.

% {\progress} 
% Этот раздел должен быть отдельным структурным элементом по
% ГОСТ, но он, как правило, включается в описание актуальности
% темы. Нужен он отдельным структурынм элемементом или нет ---
% смотрите другие диссертации вашего совета, скорее всего не нужен.

{\aim} работы является компьютерный анализ связи между конформационными свойствами мутантных белков SOD1 и наследственной формой бокового амиотрофического склероза на основе оценки стабильности сети водородных связей и водных мостиков.

Для~достижения были поставлены следующие {\tasks}:
\begin{enumerate}
  \item Разработать метод расчёта изменения конформационных свойств мутантных форм белка по сравнению с белком дикого типа на основе оценки стабильности сети внутримолекулярных водородных связей, водородных связей белка с молекулами воды и водных мостиков с использованием метода молекулярной динамики.
  \item Построить и исследовать статистические модели, связывающие конформационные характеристики мутантных форм белка SOD1, ассоциированных с БАС с продолжительностью жизни пациентов, носителей данных мутаций. Предсказать продолжительность жизни пациентов с мутациями в SOD1, данные по которым отсутствуют в литературе.
  \item Проанализировать распределение ключевых водородных связей и водных мостиков в пространственной структуре белка SOD1, важных для предсказания продолжительности жизни пациентов с БАС.
\end{enumerate}


{\novelty}
В диссертационной работе предложен метод оценки влияния мутаций на структуру и динамику белков, применение которого для изучения белков SOD1 и барназы позволило получить следующие результаты:
\begin{enumerate}
  \item Разработан метод расчёта изменения конформационных свойств мутантных форм белка по сравнению с белком дикого типа на основе оценки стабильности сети внутримолекулярных водородных связей, водородных связей белка с молекулами воды и водных мостиков с использованием метода молекулярной динамики.
  \item Построены регрессионные модели, связывающие конформационные характеристики мутантных форм белка SOD1, ассоциированных с БАС с продолжительностью жизни пациентов, носителей данных мутаций. Показано, что продолжительность жизни пациентов с мутациями в SOD1 имеет высокую корреляцию с конформационными свойствами мутантных форм белка. Коэффициент корреляции продолжительности жизни пациентов с оценками стабильности сети внутримолекулярных водородных связей составил 0.89 (p<0.001), со стабильностью водородных связей белка с молекулами воды -- 0.78 (p<0.001), со стабильностью водных мостиков -- 0.68 (p<0.001). Коэффициент множественной корреляции с учётом этих трёх факторов составил 0.9 (p<0.001).
  \item Показано, что в локальной дестабилизации белка SOD1 участвуют чаще те аминокислотные остатки, которые находятся в функциональной области белка. Выявлена важная роль водородных связей и водных мостиков в увеличении вероятности локальной дестабилизации его структуры.
  \item С помощью разработанного метода предсказана продолжительность жизни пациентов с мутациями SOD1, данные по которым отсутствуют в литературе. В частности, предсказана новая замена Вал94Ала,  которая может быть ассоциирована с БАС. Согласно предсказаниям, продолжительность жизни пациентов, потенциальных носителей данной мутации, будет составлять от 3.57 до 11.77 лет. Мутация Вал94Ала  может быть использована при планировании экспериментов по генотипированию пациентов с БАС.
\end{enumerate}

{\influence} Укладка белка и его пространственная структура целиком «заложена» в первичной аминокислотной последовательности. Поэтому фундаментальная ценность настоящей работы заключается в выявлении факторов, на уровне белка, которые бы объясняли связь мутации с его структурно-динамическими характеристиками. Прикладная ценность заключается в применении разработанных подходов для прогнозирования характера влияния на белок ранее неизвестных полиморфизмов в ответственных за болезнь генах. Это, в свою очередь, позволит в диагностике ориентироваться на конкретного пациента с индивидуальным набором полиморфизмов, а также выявлять подверженность его тем или иным заболеваниям, исходя из его генома. 

{\methods} Для решения поставленных задач в работе используются методы системного анализа, объектно-ориентированного программирования, математической статистики, молекулярной динамики.

{\defpositions}
\begin{enumerate}
  \item Метод выявления аминокислотных остатков, наиболее сильно связанных с фенотипическими признаками организма.
  \item Предсказания времени жизни пациентов БАС, которые являются носителями мутантных белков SOD1, сделанные на основе предложенного метода.
\end{enumerate}

\todo{{\reliability} полученных результатов обеспечивается \ldots \ Результаты находятся в соответствии с результатами, полученными другими авторами.}


{\probation}
Результаты работы докладывались и обсуждались на следующих конференциях:
\begin{enumerate}
\item The fifth International German/Russian Workshop in <<Integrative Biological Pathway Analysis and Simulation>> (Germany, Bielefeld, June 2--3, 2014).
\item VII Российский симпозиум с международным участием <<Белки и пептиды>> (Новосибирск, 12--17 июля 2015 г).
\item Международная конференция <<Актуальные проблемы вычислительной и прикладной математики 2015>> (Новосибирск, 19--23 октября 2015 г).
\end{enumerate}

\todo{{\contribution} Автор принимал активное участие \ldots}

%\publications\ Основные результаты по теме диссертации изложены в ХХ печатных изданиях~\cite{Sokolov,Gaidaenko,Lermontov,Management},
%Х из которых изданы в журналах, рекомендованных ВАК~\cite{Sokolov,Gaidaenko}, 
%ХХ --- в тезисах докладов~\cite{Lermontov,Management}.

\ifnumequal{\value{bibliosel}}{0}{% Встроенная реализация с загрузкой файла через движок bibtex8
    \publications\ Основные результаты по теме диссертации изложены в XX печатных изданиях, 
    X из которых изданы в журналах, рекомендованных ВАК, 
    X "--- в тезисах докладов.%
}{% Реализация пакетом biblatex через движок biber
%Сделана отдельная секция, чтобы не отображались в списке цитированных материалов
    \begin{refsection}[vak,papers,conf]% Подсчет и нумерация авторских работ. Засчитываются только те, которые были прописаны внутри \nocite{}.
        %Чтобы сменить порядок разделов в сгрупированном списке литературы необходимо перетасовать следующие три строчки, а также команды в разделе \newcommand*{\insertbiblioauthorgrouped} в файле biblio/biblatex.tex
        \printbibliography[heading=countauthorvak, env=countauthorvak, keyword=biblioauthorvak, section=1]%
        \printbibliography[heading=countauthorconf, env=countauthorconf, keyword=biblioauthorconf, section=1]%
        \printbibliography[heading=countauthornotvak, env=countauthornotvak, keyword=biblioauthornotvak, section=1]%
        \printbibliography[heading=countauthor, env=countauthor, keyword=biblioauthor, section=1]%
        \nocite{%Порядок перечисления в этом блоке определяет порядок вывода в списке публикаций автора
                Alemasov2014,Alemasov2016,%
                AlemasovIBPAS2014,AlemasovAMCA2015,AlemasovPropep2015,%
                bib1,bib2,%
        }
        \publications\ Основные результаты по теме диссертации изложены в \arabic{citeauthor} печатных изданиях, 
        \arabic{citeauthorvak} из которых изданы в журналах, рекомендованных ВАК, 
        \arabic{citeauthorconf} "--- в тезисах докладов.
    \end{refsection}
    \begin{refsection}[vak,papers,conf]%Блок, позволяющий отобрать из всех работ автора наиболее значимые, и только их вывести в автореферате, но считать в блоке выше общее число работ
        \printbibliography[heading=countauthorvak, env=countauthorvak, keyword=biblioauthorvak, section=2]%
        \printbibliography[heading=countauthornotvak, env=countauthornotvak, keyword=biblioauthornotvak, section=2]%
        \printbibliography[heading=countauthorconf, env=countauthorconf, keyword=biblioauthorconf, section=2]%
        \printbibliography[heading=countauthor, env=countauthor, keyword=biblioauthor, section=2]%
        \nocite{Alemasov2016}%vak
        \nocite{bib1}%notvak
        \nocite{AlemasovIBPAS2014}%conf
    \end{refsection}
}
