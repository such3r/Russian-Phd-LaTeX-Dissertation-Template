
{\actuality} Авторами недавней работы \cite{Bystrom2010} показано, что продолжительность жизни пациентов со дня первой диагностики заболевания отрицательно коррелирует (R = $0.78$) с потерей термостабильности мутантов, носителями которых они являлись. Однако найденная закономерность хорошо работала только для ограниченного круга мутаций, в то время как эффекты большого количества мутаций, связанных с изменением заряда аминокислотных остатков, не подчинялись построенной зависимости. Ранее в исследовании на клеточных культурах также не удалось объяснить, в общем, связь агрегации с термостабильностью, изменением заряда и др. \cite{Prudencio2009}

Большинство существующих работ \cite{Chiti2009,Sato2005,Stathopulos2003}, посвященных проблеме агрегации мутантных белков, основаны на экспериментальных данных по изменению их термодинамической стабильности. Очевидно, что помимо термостабильности, как таковой, следует также обращать внимание на детальный анализ отдельных факторов, включающих сети водородных связей, солевые мостики и другие физико-химические, структурные и конформационные характеристики белков. 
Недавнее исследование на основе метода молекулярной динамики (МД) \cite{Alder1959} показало, что структура SOD1 становится более гибкой в мутантных белках по сравнению с белком дикого типа \cite{Keerthana2015}. В частности, изменение водородных связей в мутантах Гис46Арг и Гис80Арг может приводить к неверной укладке и агрегации последних.

В рамках настоящей работы сделано предположение, что как стабилизация, так и дестабилизация структуры мутантных SOD1 может влиять на возникновение болезни \cite{Alemasov2014}. Некоторые мутации могут оказывать пространственно-распределенный эффект на физико-химические и структурные характеристики, определяющие подверженность белка к агрегации. При этом происходят локальные разнонаправленные изменения характеристик различных участков пространственной структуры белка, компенсирующие друг друга в масштабе всей структуры. Таким образом, такой интегральный показатель как термостабильность не позволяет выявить закономерности агрегации белков. В частности, <<патогенные>> мутации SOD1, оказывающие влияние на сеть водородных связей путем разрушения одних связей и возникновения других, могут повышать вероятность перехода из нативной конформации SOD1 в её <<патогенную>> форму. Стоит отметить, что на стабильность белка влияют также другие типы связей, например, такие как водородные связи с молекулами воды и водные мостики (см., например, \cite{Papoian2003,Petukhov1999}). 

Таким образом, с помощью метода МД была оценена стабильность не только внутримолекулярных водородных связей (\modelpphb{}), но также водородных связей остатков белка с молекулами воды (\modelpwhb{}) и <<водных мостиков>> (\modelwbr{}), образующихся и разрушающихся в ходе динамики белка SOD1. Эти характеристики рассчитывались как суммарное время существования каждой конкретной связи за период моделирования. В настоящем исследовании рассматривался дикий тип и расширенный список из 39 следующих мутаций белка SOD1 человека: Ала4Вал, Цис6Ала, Цис6Гли, Вал7Глу, Лей8Глн, Гли10Вал, Гли12Арг, Фен20Цис, Гли37Арг, Лей38Вал, Гли41Асп, Гли41Сер, Гис43Арг, Гис46Арг, Гис48Глн, Асп76Вал, Лей84Вал, Лей84Фен, Гли85Арг, Асн86Лиз, Ала89Вал, Асп90Ала, Гли93Арг, Вал94Ала, Глу100Гли, Асп101Асн, Сер105лей, Лей106Вал, Цис111Сер, Иле112Мет, Иле112Тре, Гли114Ала, Вал118Лей, Асп124Вал, Асп125Гис, Гли127Арг, Асн139Гис, Лей144Сер, Вал148Иле. 

На основе структурных характеристик белка SOD1 \modelpphb{}, \modelpwhb{} и \modelwbr{} построено, соответственно, три регрессионных модели, предсказывающих продолжительность жизни пациентов по структурным характеристикам мутантов SOD1. С использованием подхода bootstrap \cite{Efron1979} и ряда строгих критериев сравнения распределений показано статистически значимое отличие продолжительности жизни пациентов, предсказанной с помощью регрессий, от продолжительности жизни, ожидаемой по случайным причинам. Дополнительно были построены две комбинированные модели с использованием множественной регрессии, учитывающие параметры одновременно трёх типов связей в белке (\modelpphb{}, \modelpwhb{} и \modelwbr{}). 

Было показано, что характеристики \modelpphb{}, \modelpwhb{} и \modelwbr{}, рассчитывающиеся в построенных моделях для мутантов SOD1 и белка дикого типа, имеют корреляцию с продолжительностью жизни более сильную (R = $0.89$, p-value < $0.001$), чем корреляция, основанная на термостабильности, выявленная в работе \cite{Bystrom2010}. Построено две комбинированных регрессионных модели, опирающихся на все три характеристики SOD1 (\modelpphb{}, \modelpwhb{} и \modelwbr{}), которые имеют более высокий коэффициент корреляции, чем корреляция на основе только водородных связей внутри белка. Кроме того, в отличие от \cite{Bystrom2010}, каждая из построенных регрессионных моделей предсказывает продолжительность жизни пациентов-носителей мутантных форм белка SOD1 независимо от физико-химических свойств заменяемых и заменяющих аминокислотных остатков.

% {\progress} 
% Этот раздел должен быть отдельным структурным элементом по
% ГОСТ, но он, как правило, включается в описание актуальности
% темы. Нужен он отдельным структурынм элемементом или нет ---
% смотрите другие диссертации вашего совета, скорее всего не нужен.

{\aim} работы является компьютерный анализ связи между конформационными свойствами мутантных белков SOD1 и наследственной формой бокового амиотрофического склероза на основе оценки стабильности сети водородных связей и водных мостиков.

Для~достижения были поставлены следующие {\tasks}:
\begin{enumerate}
  \item Разработать метод расчёта изменения конформационных свойств мутантных форм белка по сравнению с белком дикого типа на основе оценки стабильности сети внутримолекулярных водородных связей, водородных связей белка с молекулами воды и водных мостиков с использованием метода молекулярной динамики.
  \item Построить и исследовать статистические модели, связывающие конформационные характеристики мутантных форм белка SOD1, ассоциированных с БАС с продолжительностью жизни пациентов, носителей данных мутаций. Предсказать продолжительность жизни пациентов с мутациями в SOD1, данные по которым отсутствуют в литературе.
  \item Проанализировать распределение ключевых водородных связей и водных мостиков в пространственной структуре белка SOD1, важных для предсказания продолжительности жизни пациентов с БАС.
\end{enumerate}


{\novelty}
В диссертационной работе предложен метод оценки влияния мутаций на структуру и динамику белков, применение которого для изучения белков SOD1 и барназы позволило получить следующие результаты:
\begin{enumerate}
  \item Разработан метод расчёта изменения конформационных свойств мутантных форм белка по сравнению с белком дикого типа на основе оценки стабильности сети внутримолекулярных водородных связей, водородных связей белка с молекулами воды и водных мостиков с использованием метода молекулярной динамики.
  \item Построены регрессионные модели, связывающие конформационные характеристики мутантных форм белка SOD1, ассоциированных с БАС с продолжительностью жизни пациентов, носителей данных мутаций. Показано, что продолжительность жизни пациентов с мутациями в SOD1 имеет высокую корреляцию с конформационными свойствами мутантных форм белка. Коэффициент корреляции продолжительности жизни пациентов с оценками стабильности сети внутримолекулярных водородных связей составил 0.89 (p<0.001), со стабильностью водородных связей белка с молекулами воды -- 0.78 (p<0.001), со стабильностью водных мостиков -- 0.68 (p<0.001). Коэффициент множественной корреляции с учётом этих трёх факторов составил 0.9 (p<0.001).
  \item Показано, что в локальной дестабилизации белка SOD1 участвуют чаще те аминокислотные остатки, которые находятся в функциональной области белка. Выявлена важная роль водородных связей и водных мостиков в увеличении вероятности локальной дестабилизации его структуры.
  \item С помощью разработанного метода предсказана продолжительность жизни пациентов с мутациями SOD1, данные по которым отсутствуют в литературе. В частности, предсказана новая замена Вал94Ала,  которая может быть ассоциирована с БАС. Согласно предсказаниям, продолжительность жизни пациентов, потенциальных носителей данной мутации, будет составлять от 3.57 до 11.77 лет. Мутация Вал94Ала  может быть использована при планировании экспериментов по генотипированию пациентов с БАС.
\end{enumerate}

{\influence} Укладка белка и его пространственная структура целиком «заложена» в первичной аминокислотной последовательности. Поэтому фундаментальная ценность настоящей работы заключается в выявлении факторов, на уровне белка, которые бы объясняли связь мутации с его структурно-динамическими характеристиками. Прикладная ценность заключается в применении разработанных подходов для прогнозирования характера влияния на белок ранее неизвестных полиморфизмов в ответственных за болезнь генах. Это, в свою очередь, позволит в диагностике ориентироваться на конкретного пациента с индивидуальным набором полиморфизмов, а также выявлять подверженность его тем или иным заболеваниям, исходя из его генома. 

{\methods} Для решения поставленных задач в работе используются методы системного анализа, объектно-ориентированного программирования, математической статистики, молекулярной динамики.

{\defpositions}
\begin{enumerate}
  \item Метод выявления аминокислотных остатков, наиболее сильно связанных с фенотипическими признаками организма.
  \item Предсказания времени жизни пациентов БАС, которые являются носителями мутантных белков SOD1, сделанные на основе предложенного метода.
\end{enumerate}

\todo{{\reliability} полученных результатов обеспечивается \ldots \ Результаты находятся в соответствии с результатами, полученными другими авторами.}


{\probation}
Результаты работы докладывались и обсуждались на следующих конференциях:
\begin{enumerate}
\item The fifth International German/Russian Workshop in <<Integrative Biological Pathway Analysis and Simulation>> (Germany, Bielefeld, June 2--3, 2014).
\item VII Российский симпозиум с международным участием <<Белки и пептиды>> (Новосибирск, 12--17 июля 2015 г).
\item Международная конференция <<Актуальные проблемы вычислительной и прикладной математики 2015>> (Новосибирск, 19--23 октября 2015 г).
\end{enumerate}

\todo{{\contribution} Автор принимал активное участие \ldots}

%\publications\ Основные результаты по теме диссертации изложены в ХХ печатных изданиях~\cite{Sokolov,Gaidaenko,Lermontov,Management},
%Х из которых изданы в журналах, рекомендованных ВАК~\cite{Sokolov,Gaidaenko}, 
%ХХ --- в тезисах докладов~\cite{Lermontov,Management}.

\ifnumequal{\value{bibliosel}}{0}{% Встроенная реализация с загрузкой файла через движок bibtex8
    \publications\ Основные результаты по теме диссертации изложены в XX печатных изданиях, 
    X из которых изданы в журналах, рекомендованных ВАК, 
    X "--- в тезисах докладов.%
}{% Реализация пакетом biblatex через движок biber
%Сделана отдельная секция, чтобы не отображались в списке цитированных материалов
    \begin{refsection}[vak,papers,conf]% Подсчет и нумерация авторских работ. Засчитываются только те, которые были прописаны внутри \nocite{}.
        %Чтобы сменить порядок разделов в сгрупированном списке литературы необходимо перетасовать следующие три строчки, а также команды в разделе \newcommand*{\insertbiblioauthorgrouped} в файле biblio/biblatex.tex
        \printbibliography[heading=countauthorvak, env=countauthorvak, keyword=biblioauthorvak, section=1]%
        \printbibliography[heading=countauthorconf, env=countauthorconf, keyword=biblioauthorconf, section=1]%
        \printbibliography[heading=countauthornotvak, env=countauthornotvak, keyword=biblioauthornotvak, section=1]%
        \printbibliography[heading=countauthor, env=countauthor, keyword=biblioauthor, section=1]%
        \nocite{%Порядок перечисления в этом блоке определяет порядок вывода в списке публикаций автора
                Alemasov2014,Alemasov2016,%
                AlemasovIBPAS2014,AlemasovAMCA2015,AlemasovPropep2015,%
                %bib1,bib2,%
        }
        \publications\ Основные результаты по теме диссертации изложены в \arabic{citeauthor} печатных изданиях, 
        \arabic{citeauthorvak} из которых изданы в журналах, рекомендованных ВАК, 
        \arabic{citeauthorconf} "--- в тезисах докладов.
    \end{refsection}
    \begin{refsection}[vak,papers,conf]%Блок, позволяющий отобрать из всех работ автора наиболее значимые, и только их вывести в автореферате, но считать в блоке выше общее число работ
        \printbibliography[heading=countauthorvak, env=countauthorvak, keyword=biblioauthorvak, section=2]%
        \printbibliography[heading=countauthornotvak, env=countauthornotvak, keyword=biblioauthornotvak, section=2]%
        \printbibliography[heading=countauthorconf, env=countauthorconf, keyword=biblioauthorconf, section=2]%
        \printbibliography[heading=countauthor, env=countauthor, keyword=biblioauthor, section=2]%
        \nocite{Alemasov2016}%vak
        %\nocite{bib1}%notvak
        \nocite{AlemasovIBPAS2014}%conf
    \end{refsection}
}
