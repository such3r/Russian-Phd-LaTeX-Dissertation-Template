\chapter{Литературный обзор} \label{chapt1}

\section{Боковой амиотрофический склероз} \label{sect_ALS}

Боковой амиотрофический склероз (БАС) -- заболевание, характеризующееся неотвратимой дегенерацией центральных и периферических моторных нейронов, приводящее к прогрессирующему параличу и возможной смерти от дыхательной недостаточности \cite{Alavi2013,Andersen2003,Drechsel2012,Haidet-Phillips2011}. Частота встречаемости заболевания составляет 1-2 случая на 100000 \cite{Alavi2013,Brotherton2012,Brown1997}.

БАС, как заболевание, был впервые описан в 1824 году Чарльзем Бэллом \cite{Rowland2001}. Большой вклад в описание патологии БАС внёс Жан Крювелье в 1852 году, а Француа Аран ещё в 1848 описал ряд случаев БАС, хотя не разделял заболевание по происхождению на нервное и мышечное \cite{Rowland2001, Marangi2015}. Сам термин <<боковой амиотрофический склероз>> был впервые упомянут в работе Жана Мартена Шарко в 1874 году \cite{Rowland2001}.

По разным данным от 1\% до 25\% случаев БАС имеют наследственную природу \cite{Abe1996,Alavi2013,Belzil2012,Eisen2008,Renton2011}. По данным секвенирования полного экзома человека, порядка тридцати генов вовлечены в заболевание БАС \cite{Cirulli2015}. Среди этих генов можно выделить следующие: \textit{SOD1}, \textit{TARDBP}, \textit{FUS}, \textit{OPTN} и \textit{VCP}. Взятые вместе, мутации в этих генах объясняют около 25\% наследственных случаев БАС \cite{Renton2011}. Кроме того, известен некодирующий регион C9orf72, в котором увеличенное количество шестинуклеотидных повторов GGGGCC также приводит к БАС \cite{Renton2011,DeJesus-Hernandez2011}.

\subsection{Некодирующий регион C9orf72} \label{subsect_C9orf72}

C9orf72 -- некодирующая область хромосомы 9, соответствующая открытой рамке считывания 72 \cite{DeJesus-Hernandez2011}. Было показано, что у здоровых людей максимальное количество повторов GGGGCC составило 23. У людей из выборки VSM-20 (по первым буквам организаций, выполнявших исследование: Vancouver, San Francisco и Mayo, семья 20), страдающих одним из заболеваний:  	лобно-височная деменция, БАС -- или обоими -- количество шестинуклеотидных повторов в C9orf72 варьировалось от 700 до 1600. На основе дополнительного анализа данных пациентов с БАС было выявлено $4.1$\% пациентов, страдающих спорадической формой заболевания и $23.5$\% пациентов -- наследственной формой, которые в то же время имели увеличенное количество повторов GGGGCC в C9orf72. 

Данный некодирующий участок хромосомы 9 изучался также другими авторами \cite{Renton2011}. При исследовании 405 финских пациентов и 497 здоровых людей обнаружено, что локус на хромосоме 9p21 объясняет до четверти всех случаев БАС и до половины всех случаев его наследственной формы \cite{Laaksovirta2010}. В последующем исследовании также был выделен участок C9orf72, количество шестинуклеотидных повторов GGGGCC в котором отделяло здоровых людей от больных БАС \cite{Renton2011}. На основе анализа данных от нескольких семей с БАС, обнаружено, что у здоровых людей повторов было меньше 20, а у больных -- больше 30. Анализ 402 финских пациентов с БАС и 478 здоровых людей выявил, что количество шестинуклеотидных повторов увеличено у $28.1$\% пациентов и $0.4$\% здоровых людей. Всего пациентов с наследственной формой БАС, у которых увеличено количество повторов  GGGGCC, оказалось $46.4$\%, а среди пациентов со спорадической формой заболевания было $21$\% с повышенным числом повторов. В среднем количество повторов у пациентов было 53, у здоровых людей -- 2. Таким образом, C9orf72 отвечает за наибольшее количество случаев заболевания БАС. 

\subsection{Ген \textit{SOD1}} \label{subsect_SOD1}

Другой наследственной причиной развития БАС являются мутации в гене \textit{SOD1}, кодирующем фермент супероксиддисмутазу-1 \cite{Bosco2010,Ivanova2014}. На данный момент известно свыше 170 мутаций этого гена, вызывающих различные формы БАС (http://alsod.iop.kcl.ac.uk/). 

Белок супероксиддисмутаза-1 (SOD1) -- гомодимер. Он катализирует реакцию превращения анионов супероксида на молекулярный кислород и пероксид водорода. Каждая из двух субъединиц белка состоит из 153 аминокислотных остатков и содержит в себе два иона: медь и цинк. Каждая субъединица имеет в своей вторичной структуре $\beta$-бочонок, составленный из 8 $\beta$-тяжей, и две функционально важных петли: электростатическую и цинк-связывающую. Также в субъединицах присутствует по одной дисульфидной связи.

\todo{\cite{Das2013}}

\subsection{Ген \textit{TARDBP}} \label{subsect_TARDBP}

Мутации в белке TDP-43, кодируемом геном \textit{TARDBP} также приводят в $1$\% случаев к заболеванию БАС, как наследственной, так и спорадической формы \cite{Corrado2009}. Заметные включения, содержащие убиквитин, как считается, являются признаком этого заболевания \cite{Kabashi2008,VanDeerlin2008}. В то же время среди белков, составляющих агрегат главный компонент -- TDP-43 \cite{Sreedharan2008,Kabashi2008,Kuhnlein2008}. TDP-43 -- РНК-связывающий и ДНК-связывающий белок, имеющий множество функций в клетке. Изначально TDP-43 был классифицирован, как репрессор транскрипции, который связывается с TAR-элементом ВИЧ. Из-за своего молекулярного веса он был назван TDP-43 \cite{VanDeerlin2008}. Белок TDP-43 участвует в регуляции экспрессии генов и сплайсинге, присутствует в комплексе, который производит сплайсинг гена \textit{CFTR} и, вероятно, также участвует в биогенезе микро-РНК, апоптозе и клеточном делении \cite{Sreedharan2008,VanDeerlin2008}. Известно, что накопление гиперфосфорилированных фрагментов TDP-43 в перикарии нейронов у пациентов с БАС связано со значительной потерей TDP-43 в ядре \cite{Sreedharan2008}. За исключением одной, все известные мутации TDP-43 сосредоточены в C-терминальной части белка -- глицин-богатом районе, который, вероятно, участвует в связывании с другими белками, включая разнообразные рибонуклеопротеины \cite{Corrado2009}. Кроме того, подтверждено наличие в образцах мозга пациентов с БАС меньшей фосфорилированной части (25 кДа) C-концевого фрагмента TDP-43, а также крупных убиквитинированных агрегатов, включающих данный белок \cite{Corrado2009}. Обнаружено, что TDP-43 связывается с высокой аффинностью с убиквилином-2 (UBQLN2) -- подобным убиквитину белком семейства UBQLN \cite{Cassel2013}. Мутации в UBQLN2 известны в качестве генетического маркера доминантного X-сцепленного варианта возникновения БАС. Вместе с тем UBQLN2 усиливает выведение, как TDP-43, так и C-концевых фрагментов, то есть может влиять на цитотоксичность \cite{Cassel2013}.

\subsection{Ген \textit{FUS}} \label{subsect_FUS}

\subsection{Ген \textit{OPTN}} \label{subsect_OPTN}

%\newpage
%============================================================================================================================

\section{Агрегация белков} \label{sect_aggregation}

Эукариотические клетки -- сложные структуры, способные разграничивать происходящие внутри биохимические реакции в пространстве и времени. Ключевым механизмом такого разграничения является разделение внутриклеточного пространства на функциональные области. Один из хорошо известных примеров таких разделённых областей -- клеточные органеллы, отделённые от внешнего пространства внутриклеточными мембранами, действующими, как физический барьер. Последние исследования открыли также немембранный способ разграничения реакций -- так называемое <<расслоение жидкостей>> (liquid demixing) -- разделение различных растворов, находящихся в жидкой фазе \cite{Aguzzi2016}. Этот способ, по-видимому, используется клеткой для динамической перестройки внутриклеточного пространства. Например, когда требуется временно провести некоторую реакцию в ограниченной от других реакций области. Так, для функционально неупорядоченных белков (от англ. Intrinsically disordered proteins) вероятность фазового разделения должна быть довольно высокой, поскольку данный тип белков из-за своей структурной гибкости способен создавать множественные взаимодействия с другими белками.

Известны примеры функционального расслоения жидкостей в эукариотических клетках: ядрышко, P-гранулы (околоядерные РНК-гранулы), стрессовые РНК-гранулы, тельца Кахаля, <<ядерные пятна>> (paraspeckles), P-тела (processing (P) bodies) \cite{Aguzzi2016}. Фактически, механизм разделения жидких фаз является универсальным, в частности, для образования частиц рибонуклеопротеинов. Другими примерами немембранных многобелковых комплексов являются центросомы, сигнальные молекулы (мембранные рецепторы) и ядерные поры.

В рамках данной парадигмы разделения фаз выделяются три различных состояния, в которых могут находиться молекулы в клетке: рассеянное (газообразное), жидкое, твёрдое \cite{Aguzzi2016}. Применительно к белкам, газообразное состояние соответствует раствору, в котором белки находятся отдельно друг от друга; жидкое состояние -- растворимым белковым комплексам, отделённым <<расслоением жидкости>>; твёрдое -- нерастворимым белковым агрегатам.

Как уже упоминалось, белки, находящиеся, в <<жидкой фазе>> имеют большое количество функциональных взаимодействий засчёт конформационной подвижности присутствующих в них неупорядоченных областей и областей с низкой сложностью аминокислотного состава (low complexity regions). При этом, известно, что белки с более протяжёнными неупорядоченными участками имеют меньшее время существования в клетке \cite{VanderLee2014}. Для белков, не подверженных и подверженных агрегации наблюдается аналогичная закономерность: чем более белок подвержен агрегации, тем более быстрый <<круговорот>> в клетке он имеет и тем более низкая его концентрация поддерживается \cite{Gsponer2012}.

\todo{С этой точки зрения агрегация ряда белков, которая приводит к различным заболеваниям человека, должна возникать из-за нарушения механизма деградации этих белков или механизма поддержания их низкой концентрации.}

\todo{С этой точки зрения белки с мутацией, которая могла бы препятствовать нормальной деградации должны существовать дольше и, наряду с приобретённым в связи с мутацией новым, нежелательным взаимодействием или функцией, тем самым, могут образовывать более стабильные белковые агрегаты.}

\todo{Накопление в нервных клетках убиквитинированных включений является признаком того, что протеасома более не справляется с переработкой повреждённых белков \cite{Sreedharan2008}.}

\todo{\cite{MadanBabu2016}}

\section{Белок SOD1 и его мутации} \label{sect_SOD1}

На молекулярном уровне мутации приводят к различным изменениям в структуре белка SOD1. Одним из следствий мутации белков SOD1 является образование внутриклеточных агрегатов \cite{Bruijn1998,Deng2006}. Предложено большое количество объяснений агрегации SOD1 \cite{Ross2004}. В соответствии с одной из этих гипотез, причиной агрегации мутантных SOD1 становится снижение их стабильности при удалении ионов металлов \cite{Bourassa2014}. Другой вероятной причиной являются посттрансляционные модификации SOD1, выделенных из тканей человека, которые снижают стабильность димера \cite{Dokholyan2015}.

\section{Конформационные свойства белков} \label{sect_proteins_features}

\section{Компьютерные методы анализа белков} \label{sect_computational_methods}

\subsection{Метод молекулярной динамики} \label{sect_methods_MD}

Важнейшим средством теоретического исследования структуры, динамики и термодинамических свойств комплексов биомакромолекул является метод молекулярной динамики (МД), который является методом компьютерного моделирования, позволяющим в течение заданного периода времени проследить эволюцию системы взаимодействующих атомов с помощью численного интегрирования уравнений движения \cite{Alder1959,Gibson1960}. Метод МД широко используется для решения задач исследования термостабильности белков, конформационных переходов, транспорта молекул, белкового фолдинга и прочих. К сожалению, разброс масштабов различных физических явлений огромен, от $10^{-8}$ сек (времена движения доменов) до 1-10 сек (время самоорганизации белков и нуклеиновых кислот), и находится вне возможностей современной МД. По этой причине исследования конформационных переходов макромолекул в настоящее время сделаны только для систем малых белков и пептидов \cite{Zhou2002,Nymeyer2003}, а для больших систем, размером более 1 млн. атомов, траектории исследовались в ограниченном временном интервале не превышающем 50 нс \cite{Freddolino2006}. 

Метод молекулярной динамики состоит в численном интегрировании уравнений движения Ньютона для системы взаимодействующих частиц (атомов), находящихся в некоторых начальных условиях (заданы начальные координаты и скорости каждого атома):

\[
F_i = -\frac{\partial U(r_1,\ldots,r_N)}{\partial r_i}
\]

где i = 1 \ldots N,--координаты атома i; N--количество атомов.

Движение атомов описывается уравнениями:

\begin{equation}
\label{eq:newton_motion}
\frac{d^2r_i}{dt^2}=\frac{F_i}{m_i}
\end{equation}

где $m_i$--масса атома i.

Типичное МД-исследование предполагает расчёт траектории развития системы во времени и её изучение на предмет интересующих свойств. При этом точность расчётов зависит от выбранной функции потенциальной энергии U, форма которой вместе с необходимыми для её использования параметрами составляют силовое поле. 

\subsubsection{Силовое поле} \label{sect_methods_forcefield}

В соответствии с квантовой теорией, для небольших молекул энергия системы может быть определена методами квантовой химии. Но при числе атомов более десяти задача становится трудно вычислимой \cite{Shaitan2006}. Поэтому в МД применяются такие потенциалы, которые зависят только от положения атомов, что позволяет исследовать системы с десятками-сотнями тысяч и даже миллионами атомов.

В существующих силовых полях МД функция потенциальной энергии разбивается на сумму вкладов от различных типов взаимодействий: $U = U_b+U_\theta+U_\Phi+U_{el}+U_{LJ}$. Среди них взаимодействия:

\begin{enumerate}

\item Ближние

	\begin{enumerate}

	\item Связующие

		\begin{enumerate}

		\item растяжение связей:
		
		\[
		U_{b} = \frac{1}{2} \sum_b{K_{b}(r-b_{0})^{2}}
		\]
		
		где $b_0$--равновесная длина связи; $K_b$--силовая константа.
		
		\item изменение валентных углов:
		
		\[
		U_{\theta} = \frac{1}{2} \sum_\theta{K_{\theta}(\theta-\theta_{0})^{2}}
		\]
		
		где $\theta_0$--равновесное  значение угла; $K_{\theta}$- силовая константа.

		\item изменение двугранных углов:

		\[
		U_{\Phi} = \sum_{\Phi}{K_{\Phi}[cos(n\Phi-\delta)+1]}
		\]
		
		где $n$--кратность торсионного барьера; $\delta$--сдвиг фазы; $K_{\Phi}$--константа, определяющая высоту потенциальных барьеров двугранных углов.

		\end{enumerate}

	\item Несвязующие 
	
	Возникают между парами атомов, принадлежащими различным молекулам. Либо принадлежащие той же самой молекуле, но разделённые, по крайней мере, тремя связями.

		\begin{enumerate}
	
		\item электростатические:
		
		\[
		U_{el} = \sum \frac{q_{i} q_{j}}{\epsilon r_{ij}}
		\]

		где $q_i$, $q_j$--парциальные заряды на атомах; $\epsilon$--диэлектрическая проницаемость среды.
		
		\item  ван-дер-ваальсовы:
		
		\[
		U_{LJ} = \sum{\left [ \frac{A}{r_{ij}^{12}}-\frac{B}{r_{ij}^{6}} \right ]}
		\]
		
		где $A$, $B$ зависят от типов атомов $i$ и $j$; $r_{ij}$— расстояние между этими атомами.

		\end{enumerate}

	\end{enumerate}

\item  Дальние электростатические

В этом случае кулоновский потенциал разбивается на две части--ближнюю и дальнюю:

\[
U_{el} = \sum { \frac{1 - erf(\beta r_{ij})}{\epsilon r_{ij}}} q_i q_j + \sum {\frac{erf(\beta r_{ij})}{\epsilon r_{ij}}} q_i q_j
\]

Здесь дробь в первой сумме ведёт себя как $1/r$ при $r \to 0$, убывает экспоненциально при $r \to \infty $ и отвечает за ближнюю часть взаимодействий, вторая дробь стремится к $1/r$ при $r \to \infty $ и отвечает за дальнюю часть. $\beta$--параметр, определяющий относительный вклад между прямой суммой и суммой в обратном пространстве (см. раздел \ref{subsubsect_methods_MD_concepts}), $erf(x) = \frac {2}{\sqrt{\pi}} \int_0^x{e^{-t^2} dt}$--функция ошибки.

\end{enumerate}

Параметры силовых полей для потенциалов взаимодействий определяются из различных экспериментальных данных (спектральные, калориметрические, кристаллографические) и квантово-химических расчётов. Наиболее часто используются такие поля, как: AMBER \cite{Cornell1995,Kollman1996,Wang2000,Hornak2006,LindorffLarsen2010,Duan2003,Garcia2002}, CHARMM \cite{Mackerell2004,Mackerell1998,Feller2000,Foloppe2004}, GROMOS \cite{Gunsteren1996}, OPLS \cite{Jorgensen1996}, CVFF (CFF, PCFF) \cite{Hagler1974}.

\subsubsection{Основные концепции МД} \label{subsubsect_methods_MD_concepts}

В своей основе МД содержит простой принцип, но дополнительные приёмы позволяют превратить этот принцип в практически полезный инструмент.

Первый шаг алгоритма моделирования предполагает расчёт сил, действующих на каждый из атомов, используя силовое поле. С изменением положения атома меняются и силы, поэтому исходные уравнения движения не могут быть разрешены аналитически--их интегрируют численно.

Интегрирование разделяется на небольшие шаги, отличающиеся между собой по времени на $\Delta t$. После того как силы рассчитаны для текущей (на время $t$) конфигурации атомов, следующим шагом является генерация новой конфигурации на момент времени $t+\Delta t$, используя уравнение \eqref{eq:newton_motion}. Координаты атомов приблизительно вычисляются с помощью ряда Тейлора:

\[
r(t+\Delta t) = r(t) + \Delta t {\frac{d}{dt}} r(t) + { \frac{(\Delta t)^2}{2!} } {\frac{d^2}{dt^2}} r(t) + \ldots
\]

В настоящий момент при моделировании МД наиболее всего распространён метод интегрирования Верле \cite{Verlet1967}. Для координат и скоростей он записывается:

\[
r_i(t + \Delta t) \approx -r_i(t - \Delta t) + 2r_i(t) + \Delta t^2 {\frac{F_i}{m_i}}
\]

\[
v_i(t) \approx {\frac{1}{2 \Delta t}} [r_i(t + \Delta t) - r_i(t - \Delta t)]
\]

Одной из модификаций этого алгоритма является leap-frog схема \cite{Hockney1974}:

\[
r_i(t + \Delta t) \approx r_i(t) + \Delta t v_i(t + \frac{\Delta t}{2})
\]

\[
v_i(t + \frac{\Delta t}{2}) \approx v_i(t - \frac{\Delta t}{2}) + \Delta t \frac{F_i}{m_i}
\]

При моделировании макромолекул шаг интегрирования выбирается равным десятой доли самого короткого периода движения. Самые быстрые движения--растяжения связей, особенно соединённых с водородом. C-H колебания имеют период около 10 фс (1 фс = $10^{-15}$ сек), что даёт шаг интегрирования порядка 1 фс.

Начальные условия для координат и скоростей задаются по-разному. Координаты атомов выбираются в соответствии с геометрией и структурой моделируемой молекулярной системы. Например, координаты атомов белков, могут быть получены с помощью рентгеноструктурного анализа или метода ЯМР. Начальные скорости же задаются с помощью генератора случайных чисел, имеют максвелловское распределение и соответствуют выбранной температуре.

Обычно размер модели и самой моделируемой системы не совпадают. Сложно представить, например, моделирование целой мембраны, которое бы могло быть проведено в достаточно короткий срок. Для решения этой проблемы используются периодические граничные условия. Молекулярная система помещается в <<ящик>>, окружённый со всех сторон своими виртуальными образами. Таким образом, атом, который пересёк границу ящика появляется с противоположной его стороны.

Другой проблемой является расчёт парных, несвязующих взаимодействий. В принципе, парные взаимодействия рассчитываются для каждой пары атомов в системе, но это приводит к числу пар, равному квадрату числа атомов. С другой стороны, закон распространения несвязующих взаимодействий таков, что они быстро убывают с ростом расстояния. Поэтому часть пар не вносят существенного вклада в энергию. Такие пары можно исключить из расчёта, введя так называемый радиус отсечения. Пары, атомы которых расположены дальше друг от друга, чем установленный радиус отсечение в расчёте не участвуют. Значения устанавливаемых радиусов обрезания обычно лежат в промежутке от 8 до 12~\AA.

Применение радиуса отсечения при расчёте дальних электростатических взаимодействий может быть не достаточно точным. Полная электростатическая энергия представляет собой сумму для N атомов в решётке (nx, ny, nz) = n:

\[
U_{el} = \frac{f}{2} \sum_{n_x}\sum_{n_y} \sum_{n_z} \sum_{i}^{N}\sum_{j}^{N} {\frac{q_i q_j}{r_{ij,\bold n }}}
\]

где $r_{ij,\bold n}$--расстояние между зарядами (но не между ближайшими образами). 

Однако эта сумма сходится медленно. В этом случае учёт взаимодействий типа заряд-заряд должен быть с использованием таких методов как, например, метод суммирования Эвальда \cite{Ewald1921}, метод Particle-Mesh Ewald (PME) \cite{Darden1993,Essmann1995}, метод Particle-Particle Particle-Mesh (PPPM или P${}^3$M) \cite{Hockney1981,Luty1995} или метод Isotropic Periodic Sum (IPS) \cite{Wu2005}. Эти алгоритмы производят полное суммирование сил между бесконечным количеством образов атомов системы (из-за периодичных граничных условий) в обратном пространстве Фурье, что улучшает сходимость. Метод суммирования Эвальда имеет сложность порядка $N^{3/2}$, PME и PPPM имеют трудоёмкость порядка $N\log N$. Ключевое отличие метода IPS от методов, использующих решётку (например, PME) в форме и распределении образов частиц. В PME образы частиц идентичны образам частиц, получаемым в периодических граничных условиях и дискретно расположены в узлах решётки. В IPS же образы частиц не совпадают с реальными, что означает, что они не существуют реально. И, в то же время, «воображаемые» образы распределены изотропным и периодическим образом вокруг частицы. Для полностью гомогенных систем IPS-образы распределены одинаково во всех трёх измерениях. Для большинства типов потенциалов существует аналитическое решение в трёхмерном IPS.

МД даёт микроскопическое описание развития системы. Чтобы перейти от микроскопического описания к макроскопическому приходится использовать статистическую механику. Среди макроскопических параметров--температура (T), давление (P), объём (V), число частиц (N), внутренняя энергия (E), химический потенциал ($\mu$).

Микроскопическое состояние системы описывается положением каждого атома $r$ и импульсом $p$ \cite{Gibbs1946}. Таким образом, чтобы представить систему в обобщённом пространстве потребуется указать 6 параметров (для трёх измерений; 3 компоненты координаты, 3--импульса). Подобное обобщённое пространство называется фазовым пространством. Мгновенное состояние системы можно представить в виде точки в фазовом пространстве. Соответственно, развитие системы--в виде множества точек--траектории.

Различают несколько типов совокупностей состояний системы, удовлетворяющих определённым условиям (ансамблей), в которых может проходить моделирование МД:

\begin{enumerate}
\item Микроканонический (NVE). Постоянными являются: число атомов (N), объём (V), и энергия (E).
\item Канонический (NVT). Постоянные: число атомов (N), объём (V) и температура (T).
\item Изобарно-изотермический (NPT). Постоянные: число атомов (N), давление (P) и температура (T).
\item Большой канонический ($\mu$VT). Постоянные: химический потенциал ($\mu$), объём (V) и температура (T).
\end{enumerate}

Термодинамические свойства могут рассчитываться МД путём усреднения по времени. В эксперименте же усреднение происходит по всему набору состояний, которые принимает система, то есть--усреднение по ансамблю. Это несоответствие разрешается с помощью эргодической гипотезы \cite{Ulenbek1965}, которая говорит, что среднее по времени равно среднему по ансамблю. Ключевой момент этого равенства--в идее того, что за время движения системы в фазовом пространстве она успевает посетить все точки этого пространства, то есть принять все возможные состояния. Фактически же для моделирования этого достичь невозможно, поскольку для систем, которые обычно моделируются фазовое пространство огромно. Но, используя различные методы становится возможным генерировать достаточно представительные наборы состояний, которые дают довольно точные результаты.

Мгновенная температура системы определяется через среднюю кинетическую энергию входящих в систему частиц:

\[
T(t) = {\frac{1}{3 N k_B}} \sum_{i=1}^N {m_i v_i^2(t)}
\]

где $N$--число атомов; $k_B$--константа Больцмана; $m_i$--масса атома $i$; $v_i$--его скорость. 

Эффективная температура системы определяется как среднее по времени: $T = \left \langle T \right \rangle$ в течение достаточно длинного временного интервала.

При моделировании обычно возникает потребность установить определённую температуру системы. Это делается с помощью установки скоростей атомов с гауссовым распределением. Однако, в процессе моделирования из-за ошибок округления температура может изменяться. Поэтому её корректируют, используя схему Берендсена \cite{Berendsen1984} или более расширенную--Нозе-Гувера \cite{Nose1984,Hoover1985}, а также модифицированную схему Берендсена \cite{Bussi2007}.

Алгоритм Берендсена предполагает соединение молекулярной системы с термостатом, который поставляет в систему или выводит из неё тепло, если потребуется. Температура системы корректируется через:

\[
{\frac{dT(t)}{dt}} = {\frac{1}{\tau}} (T_0 - T(t))
\]

где $\tau$--временная константа, определяющая величину коррекции; $T_0$--температура термостата.
В соответствии с этим соотношением, скорости масштабируются с параметром $\lambda$:

\[
\lambda = \left [1 + {\frac{\Delta t}{\tau_T}} \left ( {\frac{T_0}{T(t)}} - 1 \right ) \right ]^{\frac{1}{2}}
\]

Недостатком этого метода является неравномерное распределение энергии по степеням свободы. Разные части системы принимают различные температуры, что приводит к проблеме \cite{Harvey1998}. 

Использование более сложного термостата Нозе-Гувера предполагает его введение в саму систему, предоставляя дополнительную степень свободы. Однако, это также не приводит к физическому распределению энергии по степеням свободы \cite{Golo2004}.

Модифицированная схема Берендсена (или V-rescale термостат) добавляет случайный множитель к генерируемым скоростям, что позволяет получать корректное распределение для кинетической энергии.

Расчёт давления в МД соответствует следующему определению. Для <<ящика>> со стороной $L$ с внедрённой в него плоской поверхностью площадью $A = L^2$, ориентированной перпендикулярно оси $x$, имеем давление на эту поверхность:

\[
P_x=\frac{F_x}{A}
\]

Из второго закона Ньютона это соотношение переписывается в виде:

\[
P_x = {\frac{1}{A}} {\frac{d(m v_x)}{dt}}
\]

Таким образом, давление представляет собой <<количество>> импульса, проходящего через заданную площадь за единицу времени. Этот <<поток импульса>> состоит из двух частей: импульса атомов при пересечении площадки за время $dt-P_m$ и импульса от действия сил на атомы, лежащие по разные стороны от площадки $P_f$. 

\[
P_x=P_{mx}+P_{fx}
\]

В соответствии с молекулярно-кинетической теорией в случае идеального газа давление $P_m$ определяется, как:

\[
\langle P_m \rangle = \frac{2 N}{3 V} \langle E_k \rangle
\]

где $V$--объём; $N$--число атомов; $E_k$--кинетическая энергия в расчёте на один атом; угловые скобки означают усреднение по времени.

Давление, обусловленное действием сил между противоположными относительно воображаемой площадки, Pfx, предполагая, что эти силы попарно суммируются: 

\[
P_{fx} = \frac{1}{A} \sum_i \sum_j {F_{ij} \cdot x}
\]

где $x$--единичный вектор в положительном направлении x; $i$ пробегает по всем атомам одной стороны площадки, $j$--по атомам другой её стороны.

Усредняя по всем возможным положениям воображаемой площадки вдоль оси x, а также учитывая аналогичные выражения для площадок, расположенных перпендикулярно осям y и z, имеем полный вклад:

\[
\langle P_f \rangle = \frac{1}{3 V} \left \langle \sum_i \sum_j {F_{ij} \cdot r_{ij}} \right \rangle
\]

Сложив выражения для $P_m$ и $P_f$ получим выражение для давления в зависимости от мгновенных скоростей, координат и сил в системе:

\[
P = \frac{2 N}{3 V} E_k + \frac{1}{3 V} \sum_i \sum_j {F_{ij} \cdot r_{ij}}
\]

Большинство реальных экспериментов проводятся при постоянных температуре и давлении, что делает естественным стремление проводить моделирование в NPT-ансамбле, чтобы иметь возможность напрямую сравнивать результаты моделирования с экспериментальными данными. Методы установки давления аналогичны методам для установки температуры--система может быть соединена с баростатом, который будет поддерживать постоянное давление, регулируя объём ящика моделирования множителем $\mu$:

\[
\mu = 1 - \kappa {\frac{\Delta t}{\tau_p}} (P - P_0)
\]

где $\kappa$--коэффициент изотермической сжимаемости; $\tau_p$--константа релаксации; $P_0$--давление баростата; $P$--мгновенное давление на момент времени $t$; $\Delta t$--временной шаг.

Масштабирование объёма со множителем $\mu$ эквивалентно масштабированию координат со множителем. Причём множители могут быть неодинаковыми для масштабирования по различным осям. Более сложный метод поддержания давления реализуется с помощью алгоритма Паринелло-Рамана \cite{Parrinello1981}, который подобен термостату Нозе-Гувера.

В процессе моделирования система включает движения различных частот, наивысшие из которых принадлежат колебаниям длин связей. Именно наивысшие частоты определяют минимальный временной шаг: чем выше частота, тем меньше шаг--что влияет на конечное время моделирования. Кроме того, наибольшее влияние на конформацию молекулы оказывают низкочастотные движения. Поэтому идея <<удаления>> высокочастотных перемещений атомов вполне закономерна. Например, можно сделать связи жёсткими, ограничивая длины связей без ущерба точности вычислений. Существует ряд методов, позволяющих ввести так называемые ограничения в МД, один из которых--метод SHAKE \cite{Ryckaert1977}.

В методе SHAKE на атомные координаты накладываются голономные (зависящие только от координат) ограничения:

\[
\sigma(r_1, r_2, \ldots r_N) = 0
\]

где $N$--число атомов с количеством степеней свободы $3N$.

Если есть $k$ голономных ограничения, то общее число степеней свободы составит $3N-k$. В соответствии с этими ограничениями на атомы действуют два типа сил: молекулярные (которые присутствовали в системе до введения ограничений), а также силы, обусловленные ограничениями. Последний тип сил выглядит так:

\[
G_i = - \sum_{k=1}^K {\lambda_k \frac{\partial \sigma_k}{\partial r_i}}
\]

где $K$--количество ограничений; $\lambda_k$--множители Лагранжа.

Полная сила для атома $i$ записывается в этом случае следующим образом:

\[
F_i^{'} = F_i + G_i = - {\frac{\partial}{\partial r_i}} \left ( U(r) + \sum_{k=1}^K {\lambda_k \sigma_k} \right )
\]

Разрешая набор множителей Лагранжа, алгоритм может столкнуться с ситуацией, когда одно ограничение не позволяет удовлетворить другое. Поэтому требуется совершение нескольких итераций для того, чтобы удовлетворить все ограничения (с некоторым допуском). Похожий на SHAKE алгоритм--SETTLE \cite{Miyamoto1992}--используется для превращения молекул воды в жёсткие структуры. Помимо перечисленных, например, в пакете программ GROMACS используется метод LINCS \cite{Hess1997}, который <<сбрасывает>> длины связей на их начальные значения.

\subsection{Метод эластичных сетей} \label{subsect_methods_EN}

Метод МД для полноатомных моделей белка ограничен в своей применимости для получения протяжённых по времени моделирования траекторий. Это, в первую очередь, вызвано трудоёмкостью вычислений. Например, для того, чтобы получить траекторию МД изучаемой структуры белка SOD1 протяжённостью 50 нс, было затрачено 90 часов работы одного вычислительного узла высокопроизводительного кластера, оснащённого тремя ускорителями Tesla. Отслеживание конформационных переходов в структуре белка требуют значително большего времени моделирования--от микросекунд до миллисекунд \cite{Ding2008}. Методом МД моделирование полноатомных представлений белка потребовали бы до десятков лет расчётов. 

По этой причине для решения таких задач часто используются упрощённые модели белков, такие как, например, крупнозернистые (coarse grained) модели \cite{Rudd1998} или представления структуры белка в виде эластичной сети (elastic networks) \cite{Tirion1996}. Суть крупнозернистых приближений структуры белка состоит в том, чтобы заменить целую группу атомов одной частицей. Это позволяет снизить количество степеней свободы всей молекулярной системы и, таким образом, перейти на более крупный масштаб динамики белка. В эластичных моделях количество атомов остаётся неизменным, но потенциалы взаимодействия между ними заменяются на более простые, например, однопараметрический гармонический потенциал (аналог пружины).

Мода колебаний

Коллективность

Какие методы ещё есть? 
Для чего использовались? 
Что обнаружено? 
Какие результаты?

