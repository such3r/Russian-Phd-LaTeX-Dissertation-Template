\chapter{Боковой амиотрофический склероз} \label{chapt1}

\section{Боковой амиотрофический склероз} \label{sect_ALS}

Боковой амиотрофический склероз (БАС) -- заболевание, характеризующееся неотвратимой дегенерацией центральных и периферических моторных нейронов, приводящее к прогрессирующему параличу и возможной смерти от дыхательной недостаточности \cite{Alavi2013,Andersen2003,Drechsel2012,Haidet-Phillips2011}. Частота встречаемости заболевания составляет 1-2 случая на 100000 \cite{Alavi2013,Brotherton2012,Brown1997}.

БАС, как заболевание, был впервые описан в 1824 году Чарльзем Бэллом \cite{Rowland2001}. Большой вклад в описание патологии БАС внёс Жан Крювелье в 1852 году, а Француа Аран ещё в 1848 описал ряд случаев БАС, хотя не разделял заболевание по происхождению на нервное и мышечное \cite{Rowland2001, Marangi2015}. Сам термин <<боковой амиотрофический склероз>> был впервые упомянут в работе Жана Мартена Шарко в 1874 году \cite{Rowland2001}.

По разным данным от 1\% до 25\% случаев БАС имеют наследственную природу \cite{Abe1996,Alavi2013,Belzil2012,Eisen2008,Renton2011}. По данным секвенирования полного экзома человека, порядка тридцати генов вовлечены в заболевание БАС \cite{Cirulli2015}. Среди этих генов можно выделить следующие: \textit{SOD1}, \textit{TARDBP}, \textit{FUS}, \textit{OPTN} и \textit{VCP}. Взятые вместе, мутации в этих генах объясняют около 25\% наследственных случаев БАС \cite{Renton2011}. Кроме того, известен некодирующий регион C9orf72, в котором увеличенное количество шестинуклеотидных повторов GGGGCC также приводит к БАС \cite{Renton2011,DeJesus-Hernandez2011}.

\subsection{Некодирующий регион C9orf72} \label{subsect_C9orf72}

C9orf72 -- некодирующая область хромосомы 9, соответствующая открытой рамке считывания 72 \cite{DeJesus-Hernandez2011}. Было показано, что у здоровых людей максимальное количество повторов GGGGCC составило 23. У людей из выборки VSM-20 (по первым буквам организаций, выполнявших исследование: Vancouver, San Francisco и Mayo, семья 20), страдающих одним из заболеваний:  	лобно-височная деменция, БАС -- или обоими -- количество шестинуклеотидных повторов в C9orf72 варьировалось от 700 до 1600. На основе дополнительного анализа данных пациентов с БАС было выявлено $4.1$\% пациентов, страдающих спорадической формой заболевания и $23.5$\% пациентов -- наследственной формой, которые в то же время имели увеличенное количество повторов GGGGCC в C9orf72. 

Данный некодирующий участок хромосомы 9 изучался также другими авторами \cite{Renton2011}. При исследовании 405 финских пациентов и 497 здоровых людей обнаружено, что локус на хромосоме 9p21 объясняет до четверти всех случаев БАС и до половины всех случаев его наследственной формы \cite{Laaksovirta2010}. В последующем исследовании также был выделен участок C9orf72, количество шестинуклеотидных повторов GGGGCC в котором отделяло здоровых людей от больных БАС \cite{Renton2011}. На основе анализа данных от нескольких семей с БАС, обнаружено, что у здоровых людей повторов было меньше 20, а у больных -- больше 30. Анализ 402 финских пациентов с БАС и 478 здоровых людей выявил, что количество шестинуклеотидных повторов увеличено у $28.1$\% пациентов и $0.4$\% здоровых людей. Всего пациентов с наследственной формой БАС, у которых увеличено количество повторов  GGGGCC, оказалось $46.4$\%, а среди пациентов со спорадической формой заболевания было $21$\% с повышенным числом повторов. В среднем количество повторов у пациентов было 53, у здоровых людей -- 2. Таким образом, C9orf72 отвечает за наибольшее количество случаев заболевания БАС. 

\subsection{Ген \textit{SOD1}} \label{subsect_SOD1}

Другой наследственной причиной развития БАС являются мутации в гене \textit{SOD1}, кодирующем фермент супероксиддисмутазу-1 \cite{Bosco2010,Ivanova2014}. На данный момент известно свыше 170 мутаций этого гена, вызывающих различные формы БАС (http://alsod.iop.kcl.ac.uk/). 

Белок супероксиддисмутаза-1 (SOD1) -- гомодимер. Он катализирует реакцию превращения анионов супероксида на молекулярный кислород и пероксид водорода. Каждая из двух субъединиц белка состоит из 153 аминокислотных остатков и содержит в себе два иона: медь и цинк. Каждая субъединица имеет в своей вторичной структуре $\beta$-бочонок, составленный из 8 $\beta$-тяжей, и две функционально важных петли: электростатическую и цинк-связывающую. Также в субъединицах присутствует по одной дисульфидной связи.

\todo{\cite{Das2013}}

\subsection{Ген \textit{TARDBP}} \label{subsect_TARDBP}

Мутации в белке TDP-43, кодируемом геном \textit{TARDBP} также приводят в $1$\% случаев к заболеванию БАС, как наследственной, так и спорадической формы \cite{Corrado2009}. Заметные включения, содержащие убиквитин, как считается, являются признаком этого заболевания \cite{Kabashi2008,VanDeerlin2008}. В то же время среди белков, составляющих агрегат главный компонент -- TDP-43 \cite{Sreedharan2008,Kabashi2008,Kuhnlein2008}. TDP-43 -- РНК-связывающий и ДНК-связывающий белок, имеющий множество функций в клетке. Изначально TDP-43 был классифицирован, как репрессор транскрипции, который связывается с TAR-элементом ВИЧ. Из-за своего молекулярного веса он был назван TDP-43 \cite{VanDeerlin2008}. Белок TDP-43 участвует в регуляции экспрессии генов и сплайсинге, присутствует в комплексе, который производит сплайсинг гена \textit{CFTR} и, вероятно, также участвует в биогенезе микро-РНК, апоптозе и клеточном делении \cite{Sreedharan2008,VanDeerlin2008}. Известно, что накопление гиперфосфорилированных фрагментов TDP-43 в перикарии нейронов у пациентов с БАС связано со значительной потерей TDP-43 в ядре \cite{Sreedharan2008}. За исключением одной, все известные мутации TDP-43 сосредоточены в C-терминальной части белка -- глицин-богатом районе, который, вероятно, участвует в связывании с другими белками, включая разнообразные рибонуклеопротеины \cite{Corrado2009}. Кроме того, подтверждено наличие в образцах мозга пациентов с БАС меньшей фосфорилированной части (25 кДа) C-концевого фрагмента TDP-43, а также крупных убиквитинированных агрегатов, включающих данный белок \cite{Corrado2009}. Обнаружено, что TDP-43 связывается с высокой аффинностью с убиквилином-2 (UBQLN2) -- подобным убиквитину белком семейства UBQLN \cite{Cassel2013}. Мутации в UBQLN2 известны в качестве генетического маркера доминантного X-сцепленного варианта возникновения БАС. Вместе с тем UBQLN2 усиливает выведение, как TDP-43, так и C-концевых фрагментов, то есть может влиять на цитотоксичность \cite{Cassel2013}.

%\newpage
%============================================================================================================================

\section{Агрегация белков} \label{sect_aggregation}

Эукариотические клетки -- сложные структуры, способные разграничивать происходящие внутри биохимические реакции в пространстве и времени. Ключевым механизмом такого разграничения является разделение внутриклеточного пространства на функциональные области. Один из хорошо известных примеров таких разделённых областей -- клеточные органеллы, отделённые от внешнего пространства внутриклеточными мембранами, действующими, как физический барьер. Последние исследования открыли также немембранный способ разграничения реакций -- так называемое <<расслоение жидкостей>> (liquid demixing) -- разделение различных растворов, находящихся в жидкой фазе \cite{Aguzzi2016}. Этот способ, по-видимому, используется клеткой для динамической перестройки внутриклеточного пространства. Например, когда требуется временно провести некоторую реакцию в ограниченной от других реакций области. Так, для функционально неупорядоченных белков (от англ. Intrinsically disordered proteins) вероятность фазового разделения должна быть довольно высокой, поскольку данный тип белков из-за своей структурной гибкости способен создавать множественные взаимодействия с другими белками.

Известны примеры функционального расслоения жидкостей в эукариотических клетках: ядрышко, P-гранулы (околоядерные РНК-гранулы), стрессовые РНК-гранулы, тельца Кахаля, <<ядерные пятна>> (paraspeckles), P-тела (processing (P) bodies) \cite{Aguzzi2016}. Фактически, механизм разделения жидких фаз является универсальным, в частности, для образования частиц рибонуклеопротеинов. Другими примерами немембранных многобелковых комплексов являются центросомы, сигнальные молекулы (мембранные рецепторы) и ядерные поры.

В рамках данной парадигмы разделения фаз выделяются три различных состояния, в которых могут находиться молекулы в клетке: рассеянное (газообразное), жидкое, твёрдое \cite{Aguzzi2016}. Применительно к белкам, газообразное состояние соответствует раствору, в котором белки находятся отдельно друг от друга; жидкое состояние -- растворимым белковым комплексам, отделённым <<расслоением жидкости>>; твёрдое -- нерастворимым белковым агрегатам.

Как уже упоминалось, белки, находящиеся, в <<жидкой фазе>> имеют большое количество функциональных взаимодействий засчёт конформационной подвижности присутствующих в них неупорядоченных областей и областей с низкой сложностью аминокислотного состава (low complexity regions). При этом, известно, что белки с более протяжёнными неупорядоченными участками имеют меньшее время существования в клетке \cite{VanderLee2014}. Для белков, не подверженных и подверженных агрегации наблюдается аналогичная закономерность: чем более белок подвержен агрегации, тем более быстрый <<круговорот>> в клетке он имеет и тем более низкая его концентрация поддерживается \cite{Gsponer2012}.

\todo{С этой точки зрения агрегация ряда белков, которая приводит к различным заболеваниям человека, должна возникать из-за нарушения механизма деградации этих белков или механизма поддержания их низкой концентрации.}

\todo{С этой точки зрения белки с мутацией, которая могла бы препятствовать нормальной деградации должны существовать дольше и, наряду с приобретённым в связи с мутацией новым, нежелательным взаимодействием или функцией, тем самым, могут образовывать более стабильные белковые агрегаты.}

\todo{Накопление в нервных клетках убиквитинированных включений является признаком того, что протеасома более не справляется с переработкой повреждённых белков \cite{Sreedharan2008}.}

\todo{\cite{MadanBabu2016}}

