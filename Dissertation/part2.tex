\chapter{Методы, используемые для определения динамических свойств структуры мутантов белка SOD1} \label{chapt2}

\section{Моделирование методом молекулярной динамики} \label{sect_MD}

\subsection{Протокол моделирования} \label{subsect_MD_protocol}

В работе для моделирования в рамках метода молекулярной динамики был применён программный комплекс AMBER 12 \cite{Salomon-Ferrer2013}. В качестве аппаратной платформы выступал гибридный высокопроизводительный кластер ЦКП <<Биоинформатика>> (http://bioinformatics.bionet.nsc.ru/), который содержит ускорители NVIDIA Tesla M2090. Для анализа структуры использовалась программа UCSF Chimera \cite{Pettersen2004}. Статистическая обработка велась с помощью языка Python 2.7 и пакетов scipy, numpy, sklearn \cite{Pedregosa2011}, scikits.bootstrap. Графика подготовлена с помощью пакета matplotlib, DOG 2.0 \cite{Ren2009} и UCSF Chimera.

При моделировании МД применён следующий набор параметров: шаг интегрирования--2 фс; радиус отсечения взаимодействий--10~\AA; температура--300 К; время моделирования--50 нс. В качестве модели воды использовалась TIP3P. Кубическая ячейка имела сторону 12~\AA и состояла из более чем 20000 молекул воды, растворённого в ней белка (PDB ID: 2V0A) из 4376 атомов. В зависимости от заряда мутантного белка в <<раствор>> добавлялось порядка 36 и 30 соответственно ионов Na+ и Cl-, что соответствовало концентрации 0.137 моль.  Для обеспечения достоверности результатов моделирования каждая траектория МД повторялась 5 раз. В итоге для 40 форм белка было получено 200 траекторий, общей протяжённостью  10~мкс.

Параметры сайта связывания атома цинка получены с помощью расчётов в рамках теории функционала плотности (DFT). Был применён обменный функционал M06 с базисным набором 6–31+G* \cite{Zhao2007} из пакета Gaussian 09 \cite{Frisch2009} с эталонной экспериментальной структурой (PDB ID: 2V0A). Далее, параметры были адаптированы для силового поля с помощью модуля MTK++/MCPB \cite{Peters2010} из пакета программ AmberTools14.

Белок для нормального функционирования в клетке требует включения в свой состав ионов меди и цинка. Известно, что количество атомов меди \~0.2/димер, а цинка \~1.5/димер \cite{Ayers2014}, что говорит о том, что моделирование может осуществляться без учёта иона меди. 

Параметры \modelpphb{}, \modelpwhb{} и \modelwbr{} были получены путём анализа траектории МД утилитой cpptraj из комплекса программ AMBER. Утилита cpptraj обнаруживает водородные связи внутри белка и между атомами белком и окружающими его молекулами воды, а также водные мостики, возникающие между аминокислотными остатками белка и возвращает время существования каждой обнаруженной связи, нормированное на длину траектории, то есть число в промежутке от 0 до 1. \todo{В случае связей между атомами белка и молекулами воды возвращаемое утилитой время существования связей может превышать 1 и достигать 3, поскольку молекулы воды могут формировать до трёх водородных связей.} Водородная связь в cpptraj определялась по межатомному расстоянию, которое не должно превышать 3.0~\AA, и величине угла, образованного связями донор-водород и водород-акцептор, который, в свою очередь, не должен превышать $135^\circ$.

\subsection{Данные по времени жизни пациентов} \label{subsect_MD_survival_data}

Данные по временам жизни пациентов, носителей известных мутаций, были взяты из базы данных ALS mutation database \cite{Yoshida2010} и работы \cite{Wang2008}. Данные по мутациям Ала89Вал \cite{Sato2005} и Гли127Арг \cite{Holmoy2010} были получены из отдельных работ. Всего были найдены сведения о 36 заменах: Ала4Вал, Цис6Ала, Вал7Глу, Лей8Глн, Гли10Вал, Гли12Арг, Фен20Цис, Гли37Арг, Лей38Вал, Гли41Асп, Гли41Сер, Гис43Арг, Гис46Арг, Гис48Глн, Асп76Вал, Лей84Вал, Лей84Фен, Гли85Арг, Асн86Лиз, Ала89Вал, Асп90Ала, Гли93Арг, Глу100Гли, Асп101Асн, Сер105лей, Лей106Вал, Иле112Мет, Иле112Тре, Гли114Ала, Вал118Лей, Асп124Вал, Асп125Гис, Гли127Арг, Асн139Гис, Лей144Сер, Вал148Иле. Но регрессии строились на основе 32 мутаций--исключая информацию о Ала89Вал, Вал118Лей, Асп124Вал и Гли127Арг--поскольку по каждой из этих мутаций есть лишь неполные либо недостоверные сведения.

\section{Моделирование методом эластичных сетей} \label{sect_EN}

\subsection{Протокол моделирования} \label{subsect_EN_protocol}

По данным GeneOntology ген \textit{SOD1} вовлечён во множество биологически процессов. В частности, в окислительный стресс, передачу нервных импульсов, регуляцию кровяного давления и другие. Мутации в SOD1 являются причиной таких заболеваний человека, как, например, глаукома и боковой амиотрофический склероз. Таким образом, исследованию влияния мутаций на структурные и конформационные свойства функциональных сайтов белка SOD1 было уделено особое внимание. 
С помощью программы FoldX \cite{Guerois2002} были внесены мутации в изучаемые белки. Далее была проведена проверка и исправление полученной структуры мутантов с целью избавиться от так называемых клэшей и исправить неоптимальные конформации. С помощью системы ElNemo \cite{Suhre2004} для каждого белка были получены эластичные сетевые модели. В рамках каждой такой модели был получен ансамбль из 11 конформаций, соответствующий 5  модам колебаний с наивысшим показателем коллективности. Таким образом, для каждого белка и изучаемых мутантов были рассчитаны 55 конформаций, представляющих крупномасштабные флуктуации структуры этих молекул. С помощью пакета программ AMBER \cite{Salomon-Ferrer2013} структура, соответствующая каждой конформации впоследствии минимизировалась с целью избежать физически недостижимых торсионных углов, длин связей и других параметров.

С помощью утилиты cpptraj из пакета программ AMBER были найдены водородные связи в каждой из 55 конформаций белков. Для каждого конкретного белка и его мутантов строилась таблица водородных связей. Для этого для каждой водородной связи рассчитывалась её относительная стабильность в имеющемся ансамбле конформаций конкретного белка (дикого типа или мутанта), как доля конформаций, в которых связь присутствовала. То есть в столбцах таблицы содержались стабильности связей по 55 конформациям конкретного белка и его мутантов. Далее все стабильности мутантов белка нормировались на стабильность белка дикого типа – из каждого столбца, соответствующего мутанту вычитался столбец стабильностей связей дикого типа.

После построения таблиц для всех изучаемых белков был проведён статистический анализ. Это было достигнуто с помощью реализованной программы на языке Python, использующей пакеты numpy, scipy и sklearn. В статистическом анализе были использованы методы: главных компонент, многомерного шкалирования, кластеризации. 

С использованием метода главных компонент, применённого к таблице стабильностей водородных связей белков, были обнаружены те связи, которые входят в компоненты, объясняющие более 80~\% дисперсии стабильности этих связей. Для этого для каждой связи рассчитан коэффициент, с которым она учитывалась в полученных главных компонентах. Строилось распределение абсолютных величин коэффициентов связей. Те связи, которые попали в долю 5~\% связей с максимальными по модулю коэффициентами, в дальнейшем отбирались для визуализации и анализа. Ожидается, что эти водородные связи значительно меняют свою стабильность в мутантах по сравнению с белком дикого типа. Таким образом, ожидается, что на области белка, содержащие отобранные водородные связи, мутации влияют наиболее сильно. То есть такие водородные связи являются критическими для структуры белка. 
Кроме того, методом affinity propagation \cite{Frey2007} проведена кластеризация мутантов каждого конкретного белка по стабильностям их водородных связей. Для мутантов внутри каждого кластера выявлены критические водородные связи--связи, которые входят в компоненты, объясняющие более 80~\% дисперсии стабильности этих связей. Ожидается, что на изменение времени жизни обнаруженных связей существенно влияют мутации, попавшие в каждый найденный кластер.

\subsection{Данные по времени жизни пациентов} \label{subsect_EN_survival_data}

Для изучения SOD1 были исследованы следующие мутации: A4V, C6G, V7E, L8Q, G10V, G12R, F20C, G37R, L38V, G41D, G41S, H43R, H46R, H48Q, D76V, L84F, L84V, G85R, N86K, A89V, D90A, G93R, E100G, D101N, S105L, L106V, I112M, I112T, G114A, D124V, D125H, G127R, N139H, L144S, V148I. Данные по времени жизни пациентов идентичны полученным в разделе \ref{subsect_MD_survival_data}.

\section{Архитектура программной реализации} \label{sect_architecture}

Разработанный программный комплекс состоит из нескольких модулей. Принципиальная схема в нотации UML отображена на рис.