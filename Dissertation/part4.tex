\chapter{Обсуждение} \label{chapt4}

\section{Ассоциация конформационных свойств мутантов белка SOD1 с боковым амиотрофическим склерозом} \label{sect_SOD1_mutations}

В настоящей работе мы предположили, что как стабилизация, так и дестабилизация структуры белка влияет на увеличение вероятности нахождения его в метастабильном «патогенном» состоянии, которое описывалось в (Ross \& Poirier, 2004). Следовательно, эти факторы могут влиять и на подверженность SOD1 агрегации. В частности, можно предположить, что стабилизация структуры происходит за счет появления новых водородных связей в структуре мутанта, отсутствующих в белке дикого типа. В свою очередь, основным фактором дестабилизации структуры является разрушение водородных связей. Вновь возникшие связи могут стабилизировать белок в «патогенной» конформации, увеличивая его способность образовывать агрегаты. На поверхности свободной энергии в пространстве конформационных состояний белка такие мутации могут привести к понижению локального минимума энергии, соответствующего «патогенной» конформации белка. Разрушение водородных связей в результате мутаций может приводить к снижению потенциального барьера на поверхности свободной энергии между нативным и «патогенным» состоянием белка, что повышает вероятность перехода между этими двумя состояниями.
Анализ изменения стабильности индивидуальных водородных связей в результате мутаций позволил нам рассчитать динамику и перестройку сети этих связей, характеристики которых имели высокую корреляцию с продолжительностью жизни пациентов. 

Ранее показано, что в процессе динамики белков происходит постоянный разрыв «первичных» водородных связей и создание альтернативных, что предотвращает резкое увеличение конформационной энтропии при повышении температуры, и, следовательно, поддерживает стабильность белка (Khechinashvili et al., 2006).  Известно, что повышение температуры давления по-разному влияет на водородные связи, образованные между сближенными и удаленными друг от друга в первичной структуре аминокислотными остатками (Nisius, Grzesiek, 2012). Как показывают результаты настоящей работы, 315 водородных связей внутри белка достоверно ($p < 0.05$) меняют свою стабильность при мутациях. Из них около 55\% связей образовано между сближенными в первичной структуре остатками (между остатками с расстоянием меньше 20 позиций). При этом изменение стабильности 295 из 315 связей положительно коррелирует со временем жизни пациентов. Отрицательную корреляцию, как оказалось, имеют 17 сближенных в первичной структуре связей, а 3 отрицательно коррелирующих связи обнаружено между удалёнными остатками (расстояние более 50 позиций). Среди последних трёх: две связи между Гис48 и Арг143 (оба в первой субъединице — цепь A в файле PDB) и одна — между Глу49 (субъединица A) и Глн153 (субъединица F).  Известно, что остаток Гис48 участвует в связывании иона меди (Strange et al., 2006), а Глу49 и Глн153 расположены на интерфейсе между субъединицами. 

Таким образом, можно сказать, что мутации в большей степени вызывают изменения в стабильности водородных связей между сближенными в аминокислотной последовательности SOD1 остатками. Интересно, что около трети из найденных 315  водородных связей образовано между удалёнными в последовательности остатками, а также между различными функциональными областями белка. Разрушение таких связей может привести к изменению конформации этих функциональных частей. Этот факт говорит о том, что мутации могут существенно влиять на выполняемую белком функцию. 
Помимо времени существования водородных связей в данной работе исследовано также влияние мутаций на время существования водных мостиков. Водные мостики образуются за счёт взаимодействия молекул воды с аминокислотными остатками белка, доступными растворителю. Подобные связи играют существенную роль в стабилизации белковой структуры (Petukhov et al., 1999), а также в ассоциации белков (Papoian et al., 2003). Всего найдено 118 связей данного типа, изменение стабильности которых достоверно коррелирует со временем жизни пациентов с БАС. В частности, продолжительность жизни пациентов коррелирует с изменением времени существования связи остатка Асн19 ($\beta$-нить 2) с Ала152 (интерфейс между субъединицами), образованной посредством молекул воды. 
В отличие от водородных связей, большая часть водных мостиков (51\%) образованы между аминокислотными остатками, удалёнными в первичной структуре. Это происходит, вероятно, потому, что водные мостики обычно в два раза длиннее, чем водородные связи. Кроме того, молекулам воды сложнее проникнуть вглубь белка. Это приводит к тому, что водные мостики образуются главным образом между остатками, расположенными на поверхности белка. Можно ожидать, что более протяжённые водные мостики являются посредниками во взаимодействии между аминокислотными остатками из функционально различных областей белка.

В ходе анализа моделей CLS и CRF было обнаружено, что между отдельными факторами P-Phb, P-Whb и Wbr во множественной регрессии существовует зависимость. В частности, коэффициент множественной корреляции в модели CLS оказался равен 0.9. В то же время корреляция с продолжительностью жизни пациентов каждого из факторов P-Phb, P-Whb и Wbr составила 0.89, 0.67 и 0. 73, соответственно. То есть добавление независимых переменных P-Whb и Wbr к P-Phb увеличило коэффициент множественной корреляции лишь незначительно. Этот эффект возник из-за того, что между одними и теми же аминокислотными остатками сформировались связи (водородные, внутри белка, с молекулами воды и водные мостики), которые оказались важными для всех трёх моделей. Другими словами, по-видимому, окружающие молекулы воды оказывают существенное влияние на динамику водородных связей внутри белка. Свидетельства в пользу этого предположения получены в работе (Tarek, Tobias, 2002), в которой на основе моделирования методом МД установлено, что мобильность молекул растворителя, окружающего белок, напрямую влияет на структурную релаксацию белка. 
Согласно нашим расчетам (см. Рис. 6) мутации в белке SOD1 оказывают стабилизирующее влияние на большую часть (94\%) всех водородных связей, учтённых в модели P-Phb. В частности, одной из стабилизированных в мутантных белках водородных связей, можно выделить связь Гис71-Лиз136. Известно, что Гис71 участвует в связывании иона цинка, который стабилизирует структуру SOD1 (Arnesano et al., 2004; Ding \& Dokholyan, 2008). Примером отрицательной корреляции между стабильностью водородных связей в мутантных белках SOD1 и продолжительностью жизни пациентов, носителей этих мутаций, может являться водородная связь Гис120-Арг143. Оба аминокислотных остатка являются фунционально важными. Гис.120 связывает ион меди, а Арг143 участвует в каталитической активности SOD1 (Muneeswaran et al., 2014).

Как видно из Рис. 6, большая часть обнаруженных на этапе анализа связей образуется между функционально значимыми в белке SOD1 аминокислотными остатками. Известно, что остатки 38-40 стабилизируют $\beta$-бочонок (H. X. Deng et al., 1995). При этом Лей38 имеет ван-дер-Ваальсовый контакт с Гис43, который, в свою очередь, связан с каталитически важным остатком Арг143. Результаты нашего моделирования (Рис. 6) хорошо согласуются с этими данными. В частности, модель P-Phb демонстрирует, что изменение времени существования водородной связи $\text{O---}\text{H}_{12}\cdots\text{N}_1$ остатка Арг143 коррелирует с продолжительностью жизни пациентов с БАС. Модель Wbr показывает структурную значимость Лей38, который посредством молекул воды связывается с остатком Асп92.

В рамках настоящей работы была промоделирована замена Вал94Ала. Вал94 образует водородную связь с Асп90 и Лей38, мутации которых, как известно, связаны с БАС. Ранее не было известно, о том, вызывает ли Вал94Ала заболевание. Замена на аланин была выбрана из-за того, что обычно
Известно, что при поиске функционально важных аминокислотных остатков часто используется аланиновый скрининг, при котором все аминокислотные остатки изучаемого белка поочерёдно заменяются на аланин (Cunningham, Wells, 1989). Как ожидается, такие мутации в наименьшей степени нарушают структуру белка, но оказывают влияние на его функцию при заменах функционально важных аминокислот. В связи с этим, нами была промоделирована замена на аланин валина в позиции 94,  который образует водородные связи с остатками Асп90 и Лей38, ассоциированными с БАС. Время жизни пациентов, носителей мутации Вал94Ала, не найдено в литературных источниках. Однако, согласно нашим предсказаниям, данная мутация может вызывать у пациентов заболевание БАС. Продолжительность жизни пациентов, носителей таких мутаций, будет составлять от 3.57 до 11.77 года по данным моделей P-Phb, P-Whb и Wbr или от 4.09 до 5.71 лет по данным комбинированных моделей CLS и CRF. 

В работе (Elam et al., 2003) описано взаимодействие между субъединицами SOD1, приводящее к формированию амилоидов. Авторы показали, что подобное взаимодействие возможно между мутантами, лишёнными ионов металла в активном сайте («apo»-форма). Ими описано, как линейное (в виде нитей), так и спиральное (в виде микротрубочек) взаимодействия субъединиц. В случае линейных нитей, белки агрегируют через участок $\beta$-бочонка (остатками 45, 87–88, 97–99) и электростатическую петлю (остатки 125–131). В случае спиральных образований--через остатки 78–81, 101, 103 обоих мономеров. Значимость этих остатков согласуется с результатами нашего моделирования (рис. 6). В частности, остатки 78–79, 81, 125, 127–129 и 131 оказались важными в модели P-Phb, остатки 80, 88, 103 и 125–129 важны в модели P-Whb, а остатки 97, 125–128 и 130–131 важны в модели Wbr. 

Другими авторами (Antonyuk et al., 2005) при изучении мутантной формы H46R белка SOD1 выделяется три области, участвующие в образовании агрегатов: электростатическая петля (остатки с 121 по 144), дисульфидная петля (остатки  49–62) и петля, содержащая сайт связывания иона цинка (остатки  63–84).  Электростатическая петля участвует в каталитической активности белка (также в (Keerthana \& Kolandaivel, 2015)). Дисульфидная--связана дисульфидной связью (через остатки 57 и 146) с $\beta$-нитью 8 (143–148) $\beta$-бочонка. Помимо этого, существуют взаимодействия и между петлями через водородные связи с остатком 124. Сам по себе остаток 124 образует водородную связь с остатком 71, который связывает атом цинка, и с остатком 46, который связывает ион меди. Такая комплексная связь авторами называется «вторичным мостом» по аналогии с «первичным мостом»--остатком 63, связывающим оба металла. Другими словами, в белке существует двойная связь между элементами электростатической и цинк-связывающей петель, что стабилизирует сайты связывания ионов металлов. В результате моделирования обнаружено (Рис. 6), что дестабилизация многих остатков в электростатической петле связана с изменением времени жизни пациентов с мутациями в SOD1. Остатки, принадлежащие петле, содержащей сайт связывания иона цинка, также выделяются моделями, как важные в смысле влияния на продолжительность жизни пациентов с БАС. Роль Гис71 (P-Phb, P-Whb, Wbr) и Асп124 (Wbr) уже обсуждалась выше и, несомненно, является значительной для стабильности белка. Эти данные согласуются с приведёнными выше сведениями о значимости перечисленных элементов вторичной структуры белка для стабильности сайтов связывания ионов металлов. 

В этой же работе (Antonyuk et al., 2005) авторы приводят сведения о дополнительных взаимодействиях, которые возникают в связи с мутацией H46R. Спиральные образования возникают из-за контактов между димерами через остатки 11, 13 и 36. Образования в виде зигзага возникают через остатки $\beta$-бочонка и электростатической петли, описанные выше (Elam et al., 2003). Образования в виде зигзага и линейных нитей могут контактировать друг с другом через остатки 109, 12–13, 37–39, 13–15, 36–39, 91–92. Построенные нами модели отметили большинство из перечисленных остатков как важные.

Интересным оказалось, что  аминокислотные остатки Асп90 и Вал94 формируют водородные связи с Лей38, замены в котором связаны с БАС. Район белка, составленный остатками 90–94 связывает  $\beta$-нить 5 (остатки 83–89) и 6 (остатки 95–101), которые, как считается, участвуют в образовании амилоидов (Banci et al., 2009). При этом модели Wbr и P-Whb, учитывающие взаимодействия с молекулами воды, (Рис. 6) подтверждают важность времени существования связей, в образовании которых участвуют атомы перечисленных остатков, для продолжительности жизни пациентов с мутациями в белке SOD1.

Известно, что $\beta$-нити 3 и 4 (а именно--остатки 21–53) белка SOD1 дикого типа, лишённого ионов металлов, а также мутантов A4V и G93R частично разупорядочены в условиях физиологических значений температуры (Durazo et al., 2009). Кроме того, для лишённых ионов металлов SOD1 дикого типа и мутантов A4V, H48Q и G93R область 117–144 дестабилизирована даже при 10~${}^\circ$C. Авторы также указывают на две области, которые являются наиболее стабильными. Первая--сформированная остатками 7–20, вторая--остатками 104–116. Последние две области, по мнению авторов, играют роль опорных точек в фолдинге SOD1. Из Рис. 6 видно, что ряд остатков, образующие водородные связи и водные мостики действительно играют важную роль в стабилизации и дестабилизации $\beta$-нитей 3 и 4. Кроме того, остатки, попадающие в электростатическую петлю, входящую в область 117–144, также выделяются моделями, как важные. Наконец, в опорных для фолдинга SOD1 областях (7–20, 104–116) также выделяются остатки, стабилизация или дестабилизация которых влияет на продолжительность жизни пациентов с БАС.

В работе (Wright et al., 2013) выделяются два района (1-30 и 90-120), имеющих особое значение для агрегации мутантных белков SOD1. По данным всех трёх моделей в этих районах стабильность водородных связей и водных мостиков коррелирует с продолжительностью жизни пациентов (Рис. 6). В этой работе (Wright et al., 2013) выделен также сайт области димеризации (Val7-Gly147-Val148), остатки которого не выделены в качестве важных построенными нами моделями. Это несоответствие может быть вызвано тем, что существуют и другие факторы, приводящие к агрегации, которые не учитываются в наших моделях.

Помимо того, что мутации в SOD1 могут повлечь за собой агрегацию данных мутантных  белков, есть сведения (Pasinelli et al., 2004), указывающие на связь мутантных SOD1 с апоптозом. В упомянутой работе говориться о прямой ассоциации SOD1, как дикого типа, так и мутантов, с анти-апоптотическим белком BCL-2. Выявлен регион BCL-2, с которым происходит связывание SOD1. При этом, SOD1 дикого типа выполняет анти-апоптотическую функцию, в то время как мутантные SOD1 могут её терять. Кроме того, BCL-2 захватывается мутантными SOD1, агрегированными в нерастворимый комплекс, что ведёт к тому, что BCL-2 теряет возможность выполнять анти-апоптотическую функцию и провоцирует гибель двигательных нейронов.