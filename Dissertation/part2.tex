\chapter{Динамические свойства структуры мутантов белка SOD1} \label{chapt2}

\section{Моделирование методом молекулярной динамики} \label{sect_MD}

\subsection{Протокол молекулярной динамики} \label{subsect_MD_protocol}

В работе для моделирования в рамках метода молекулярной динамики был применён программный комплекс AMBER 12 (Salomon-Ferrer, Case, \& Walker, 2013). В качестве аппаратной платформы выступал гибридный высокопроизводительный кластер ЦКП «Биоинформатика» (http://bioinformatics.bionet.nsc.ru/), который содержит ускорители NVIDIA Tesla M2090. Для анализа структуры использовалась программа UCSF Chimera (Pettersen et al., 2004). Статистическая обработка велась с помощью языка Python 2.7 и пакетов scipy, numpy, sklearn (Pedregosa et al. 2011), scikits.bootstrap. Графика подготовлена с помощью пакета matplotlib, DOG 2.0 (Ren et al., 2009) и UCSF Chimera.

При моделировании МД применён следующий набор параметров: шаг интегрирования--2 фс; радиус отсечения взаимодействий--10~\AA; температура--300 К; время моделирования--50 нс. В качестве модели воды использовалась TIP3P. Кубическая ячейка имела сторону 12~\AA и состояла из более чем 20000 молекул воды, растворённого в ней белка (PDB ID: 2V0A) из 4376 атомов. В зависимости от заряда мутантного белка в «раствор» добавлялось порядка 36 и 30 соответственно ионов Na+ и Cl-, что соответствовало концентрации 0.137 моль.  Для обеспечения достоверности результатов моделирования каждая траектория МД повторялась 5 раз. В итоге для 40 форм белка было получено 200 траекторий, общей протяжённостью  10 мкс.

Параметры сайта связывания атома цинка получены с помощью расчётов в рамках теории функционала плотности (DFT). Был применён обменный функционал M06 с базисным набором 6–31+G* (Zhao \& Truhlar, 2007) из пакета Gaussian 09 (Frisch et al., 2009) с эталонной экспериментальной структурой (PDB ID: 2V0A). Далее, параметры были адаптированы для силового поля с помощью модуля MTK++/MCPB (Peters et al., 2010) из пакета программ AmberTools14.

Белок для нормального функционирования в клетке требует включения в свой состав ионов меди и цинка. Известно, что количество атомов меди ~0.2/димер, а цинка ~1.5/димер (Ayers et al., 2014), что говорит о том, что моделирование может осуществляться без учёта иона меди. 

Параметры P-Phb, P-Whb и Wbr были получены путём анализа траектории МД утилитой cpptraj из комплекса программ AMBER. Утилита cpptraj обнаруживает водородные связи внутри белка и между атомами белком и окружающими его молекулами воды, а также водные мостики, возникающие между аминокислотными остатками белка и возвращает время существования каждой обнаруженной связи, нормированное на длину траектории, то есть число в промежутке от 0 до 1. \todo{В случае связей между атомами белка и молекулами воды возвращаемое утилитой время существования связей может превышать 1 и достигать 3, поскольку молекулы воды могут формировать до трёх водородных связей.} Водородная связь в cpptraj определялась по межатомному расстоянию, которое не должно превышать 3.0~\AA, и величине угла, образованного связями донор-водород и водород-акцептор, который, в свою очередь, не должен превышать $135^\circ$.

\subsection{Данные по времени жизни пациентов} \label{subsect_survival_data}

Данные по временам жизни пациентов, носителей известных мутаций, были взяты из базы данных ALS mutation database (Yoshida et al., 2010) и работы (Wang, Johnson, Agar, \& Agar, 2008). Данные по мутациям Ала89Вал (Sato et al., 2005) и Гли127Арг (Holmøy, Wilson, von der Lippe, Andersen, \& Berg-Hansen, 2010) были получены из отдельных работ. Всего были найдены сведения о 36 заменах: Ала4Вал, Цис6Ала, Вал7Глу, Лей8Глн, Гли10Вал, Гли12Арг, Фен20Цис, Гли37Арг, Лей38Вал, Гли41Асп, Гли41Сер, Гис43Арг, Гис46Арг, Гис48Глн, Асп76Вал, Лей84Вал, Лей84Фен, Гли85Арг, Асн86Лиз, Ала89Вал, Асп90Ала, Гли93Арг, Глу100Гли, Асп101Асн, Сер105лей, Лей106Вал, Иле112Мет, Иле112Тре, Гли114Ала, Вал118Лей, Асп124Вал, Асп125Гис, Гли127Арг, Асн139Гис, Лей144Сер, Вал148Иле. Но регрессии строились на основе 32 мутаций--исключая информацию о Ала89Вал, Вал118Лей, Асп124Вал и Гли127Арг--поскольку по каждой из этих мутаций есть лишь неполные либо недостоверные сведения.

\section{Моделирование методом эластичных сетей} \label{sect_EN}

\subsection{Метод эластичных сетей} \label{subsect_EN}

Метод МД для полноатомных моделей белка ограничен в своей применимости для получения протяжённых по времени моделирования траекторий. Это, в первую очередь, вызвано трудоёмкостью вычислений. Для того, чтобы получить траекторию МД изучаемой структуры белка SOD1 протяжённостью 50 нс, было затрачено 90 часов работы одного вычислительного узла высокопроизводительного кластера, оснащённого тремя ускорителями Tesla. Отслеживание конформационных изменений в структуре белка требуют значително большего времени моделирования — от микросекунд до миллисекунд (Ding \& Dokholyan, 2008). Методом МД моделирование полноатомных представлений белка потребовали бы до десятков лет расчётов. 

По этой причине для решения таких задач часто используются упрощённые модели белков, такие как, например, крупнозернистые (coarse grained) модели (Rudd, Broughton, 1998) или представления структуры белка в виде эластичной сети (elastic networks) (Tirion, 1996). Суть крупнозернистых приближений структуры белка состоит в том, чтобы заменить целую группу атомов одной частицей. Это позволяет снизить количество степеней свободы всей молекулярной системы и, таким образом, перейти на более крупный масштаб динамики белка. В эластичных моделях количество атомов остаётся неизменным, но потенциалы взаимодействия между ними заменяются на более простой, например, однопараметрический гармонический потенциал (аналог пружины).

Какие методы ещё есть? 
Для чего использовались? 
Что обнаружено? 
Какие результаты?

\section{Архитектура программной реализации} \label{sect_architecture}

Разработанный программный комплекс состоит из нескольких модулей. Принципиальная схема в нотации UML отображена на рис.